\documentclass[UTF8]{ctexart}

\usepackage{graphics}
\usepackage{indentfirst}
\usepackage{enumerate}
\usepackage{listings}
\usepackage{xcolor}

\usepackage{float}
\usepackage{graphicx}
\usepackage{subfigure}

\setlength{\parindent}{2em}
\newcommand\degree{^\circ}

\title{\heiti 一种基于多路复用器的键盘扫描方案}
\author{\kaishu 顾顺}
\date{\today}



\begin{document}
	\maketitle

	\newpage

	\tableofcontents
	
	\begin{abstract}
		\textbf{关键词: }键盘扫描方案\space 多路复用器 \space 模拟开关
	\end{abstract}

	\newpage

	\section{引言}
	\indent 本文介绍了一种基于CD4067B的键盘扫描方式,具有较高的回报率理论值,\\\\
	\indent 该键盘扫描方案具有高速、高精度、占用IO口少等优点,\\\\
	\indent 本文将分别提出两种键盘扫描方案:被动扫描方案和主动扫描方案。\\\\
	
	\newpage
	
	\section{被动扫描方案}
	\subsection{方案思路}
		\indent 采用CD4067B\space 16进1模拟开关,将每16个按键的输出通过轮询集中到一个IO口上,用MCU的GPIO操作模拟开关的地址输入。 \\\\
		\indent 该方案支持级联,只需将每个4067B的地址输入并联,分别读取输出即可。 \\\\
	
	\subsection{原理图}
		\begin{figure}[htbp]
			\centering 
			\includegraphics[width=0.75\textwidth]{im1}
			\caption{被动扫描方案原理图}
			% \label{fig:example-image}
		\end{figure}
	
	\subsection{方案优点与缺陷}
		优点:
		\begin{enumerate}[\indent(1)\space]

			\item 极高的回报率:4067B的工作频率\cite{ref1}在MHz级别\cite{ref1},因此回报率的限制就在于MCU的扫描速度,可以轻易达到kHz级。

			\item 较少的IO口:每$16n$个按键的扫描仅需$(n+4)$个IO口。

			\item 低功耗:根据官方datasheet,4067B的功耗(5V输入)典型值为$0.2\mu W$,不超过$25\mu W$\cite{ref1}。

		\end{enumerate}
		
		缺点:
		\begin{enumerate}[\indent(1)\space]

			\item 占用资源较多:4067B的输入信号全部由MCU发生,在采用低算力的MCU(同时处理按键事件,作为HID设备)时会导致资源不够,产生延迟。

		\end{enumerate}
		
		
	\newpage
	
	\section{主动扫描方案}
	\subsection{方案思路}
		\indent 为解决被动扫描方案的缺陷,该方案在其基础上,加入信号发生器NE555\cite{ref2}(双稳态工作模式)和二进制计数器CD4520B,将其输出作为4067B的地址输入。\\\\
	
	\subsection{原理图}
		\begin{figure}[htbp]
			\centering 
			\includegraphics[width=0.75\textwidth]{im2}
			\caption{被动扫描方案原理图}
		\end{figure}
	
	\newpage
	
	\section{被动扫描方案验证}
		\indent 采用TSSOP封装的CD4067BPWR,占用更小的体积。
		\begin{figure}[htbp]
			\centering 
			\includegraphics[width=0.75\textwidth]{im3}
		\end{figure}
		\begin{figure}[htbp]
			\centering 
			\includegraphics[width=0.75\textwidth]{im4}
		\end{figure}
		
		\newpage
		
		代码实现(Arduino框架下的ESP32):
		
		\begin{lstlisting}
#include <Arduino.h>

bool p[17],q[17];
int mapp[17]={3,7,4,8,12,16,15,11,14,13,10,9,6,5,2,1,0};

void setup() {
  pinMode(12,OUTPUT);
  pinMode(14,OUTPUT);
  pinMode(27,OUTPUT);
  pinMode(26,OUTPUT);

  pinMode(5,OUTPUT);
  pinMode(4,OUTPUT);
  pinMode(15,OUTPUT);
  pinMode(17,OUTPUT);


  pinMode(25,INPUT);
  for(int i=0;i<16;i++) q[i]=0;

  Serial.begin(115200);
}

bool ctc(int x){
  return x?HIGH:LOW;
}
int cnt;
void loop() {
  for(int i=0;i<16;i++){
    digitalWrite(26,ctc(i&1));
    digitalWrite(27,ctc(i&2));
    digitalWrite(14,ctc(i&4));
    digitalWrite(12,ctc(i&8));
    p[i]=digitalRead(25);
  }
  cnt=0;
  for(int i=0;i<16;i++){
    if(p[i]) cnt++;
    if(p[i]^q[i]){
      if(p[i]){
        Serial.print("down: ");
        Serial.println(mapp[i]);
      }else{
        Serial.print("up: ");
        Serial.println(mapp[i]);
      }
    }
    p[i]^=q[i]^=p[i]^=q[i];
  }
  digitalWrite(5,ctc(cnt&1));
  digitalWrite(4,ctc(cnt&2));
  digitalWrite(15,ctc(cnt&4));
  digitalWrite(17,ctc(cnt&8));
  delay(10);
}


		\end{lstlisting}
		
	\newpage
	
	\section{应用}
		\indent 本扫描方案在实际应用中具有极小的功耗和占用体积,适用于超薄便携式键盘等,也可以应用于其他键盘中。
		
	\newpage
	
	\begin{thebibliography}{99}  
		CD4067B\space datasheet:\\
		\indent https://www.ti.com/cn/lit/ds/symlink/cd4067b.pdf\\\\
		NE555\space datasheet:\\
		\indent https://www.ti.com/cn/lit/ds/symlink/sa555.pdf
	
	\end{thebibliography}
\end{document}
